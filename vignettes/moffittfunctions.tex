\documentclass[table]{article}
\usepackage{lmodern}
\usepackage{amssymb,amsmath}
\usepackage{ifxetex,ifluatex}
\usepackage{fixltx2e} % provides \textsubscript
\ifnum 0\ifxetex 1\fi\ifluatex 1\fi=0 % if pdftex
  \usepackage[T1]{fontenc}
  \usepackage[utf8]{inputenc}
\else % if luatex or xelatex
  \ifxetex
    \usepackage{mathspec}
  \else
    \usepackage{fontspec}
  \fi
  \defaultfontfeatures{Ligatures=TeX,Scale=MatchLowercase}
\fi
% use upquote if available, for straight quotes in verbatim environments
\IfFileExists{upquote.sty}{\usepackage{upquote}}{}
% use microtype if available
\IfFileExists{microtype.sty}{%
\usepackage{microtype}
\UseMicrotypeSet[protrusion]{basicmath} % disable protrusion for tt fonts
}{}
\usepackage[margin=1in]{geometry}
\usepackage{hyperref}
\hypersetup{unicode=true,
            pdftitle={Introduction to MoffittFunctions},
            pdfauthor={William (Jimmy) Fulp},
            pdfborder={0 0 0},
            breaklinks=true}
\urlstyle{same}  % don't use monospace font for urls
\usepackage{color}
\usepackage{fancyvrb}
\newcommand{\VerbBar}{|}
\newcommand{\VERB}{\Verb[commandchars=\\\{\}]}
\DefineVerbatimEnvironment{Highlighting}{Verbatim}{commandchars=\\\{\}}
% Add ',fontsize=\small' for more characters per line
\usepackage{framed}
\definecolor{shadecolor}{RGB}{248,248,248}
\newenvironment{Shaded}{\begin{snugshade}}{\end{snugshade}}
\newcommand{\AlertTok}[1]{\textcolor[rgb]{0.94,0.16,0.16}{#1}}
\newcommand{\AnnotationTok}[1]{\textcolor[rgb]{0.56,0.35,0.01}{\textbf{\textit{#1}}}}
\newcommand{\AttributeTok}[1]{\textcolor[rgb]{0.77,0.63,0.00}{#1}}
\newcommand{\BaseNTok}[1]{\textcolor[rgb]{0.00,0.00,0.81}{#1}}
\newcommand{\BuiltInTok}[1]{#1}
\newcommand{\CharTok}[1]{\textcolor[rgb]{0.31,0.60,0.02}{#1}}
\newcommand{\CommentTok}[1]{\textcolor[rgb]{0.56,0.35,0.01}{\textit{#1}}}
\newcommand{\CommentVarTok}[1]{\textcolor[rgb]{0.56,0.35,0.01}{\textbf{\textit{#1}}}}
\newcommand{\ConstantTok}[1]{\textcolor[rgb]{0.00,0.00,0.00}{#1}}
\newcommand{\ControlFlowTok}[1]{\textcolor[rgb]{0.13,0.29,0.53}{\textbf{#1}}}
\newcommand{\DataTypeTok}[1]{\textcolor[rgb]{0.13,0.29,0.53}{#1}}
\newcommand{\DecValTok}[1]{\textcolor[rgb]{0.00,0.00,0.81}{#1}}
\newcommand{\DocumentationTok}[1]{\textcolor[rgb]{0.56,0.35,0.01}{\textbf{\textit{#1}}}}
\newcommand{\ErrorTok}[1]{\textcolor[rgb]{0.64,0.00,0.00}{\textbf{#1}}}
\newcommand{\ExtensionTok}[1]{#1}
\newcommand{\FloatTok}[1]{\textcolor[rgb]{0.00,0.00,0.81}{#1}}
\newcommand{\FunctionTok}[1]{\textcolor[rgb]{0.00,0.00,0.00}{#1}}
\newcommand{\ImportTok}[1]{#1}
\newcommand{\InformationTok}[1]{\textcolor[rgb]{0.56,0.35,0.01}{\textbf{\textit{#1}}}}
\newcommand{\KeywordTok}[1]{\textcolor[rgb]{0.13,0.29,0.53}{\textbf{#1}}}
\newcommand{\NormalTok}[1]{#1}
\newcommand{\OperatorTok}[1]{\textcolor[rgb]{0.81,0.36,0.00}{\textbf{#1}}}
\newcommand{\OtherTok}[1]{\textcolor[rgb]{0.56,0.35,0.01}{#1}}
\newcommand{\PreprocessorTok}[1]{\textcolor[rgb]{0.56,0.35,0.01}{\textit{#1}}}
\newcommand{\RegionMarkerTok}[1]{#1}
\newcommand{\SpecialCharTok}[1]{\textcolor[rgb]{0.00,0.00,0.00}{#1}}
\newcommand{\SpecialStringTok}[1]{\textcolor[rgb]{0.31,0.60,0.02}{#1}}
\newcommand{\StringTok}[1]{\textcolor[rgb]{0.31,0.60,0.02}{#1}}
\newcommand{\VariableTok}[1]{\textcolor[rgb]{0.00,0.00,0.00}{#1}}
\newcommand{\VerbatimStringTok}[1]{\textcolor[rgb]{0.31,0.60,0.02}{#1}}
\newcommand{\WarningTok}[1]{\textcolor[rgb]{0.56,0.35,0.01}{\textbf{\textit{#1}}}}
\usepackage{graphicx,grffile}
\makeatletter
\def\maxwidth{\ifdim\Gin@nat@width>\linewidth\linewidth\else\Gin@nat@width\fi}
\def\maxheight{\ifdim\Gin@nat@height>\textheight\textheight\else\Gin@nat@height\fi}
\makeatother
% Scale images if necessary, so that they will not overflow the page
% margins by default, and it is still possible to overwrite the defaults
% using explicit options in \includegraphics[width, height, ...]{}
\setkeys{Gin}{width=\maxwidth,height=\maxheight,keepaspectratio}
\IfFileExists{parskip.sty}{%
\usepackage{parskip}
}{% else
\setlength{\parindent}{0pt}
\setlength{\parskip}{6pt plus 2pt minus 1pt}
}
\setlength{\emergencystretch}{3em}  % prevent overfull lines
\providecommand{\tightlist}{%
  \setlength{\itemsep}{0pt}\setlength{\parskip}{0pt}}
\setcounter{secnumdepth}{5}
% Redefines (sub)paragraphs to behave more like sections
\ifx\paragraph\undefined\else
\let\oldparagraph\paragraph
\renewcommand{\paragraph}[1]{\oldparagraph{#1}\mbox{}}
\fi
\ifx\subparagraph\undefined\else
\let\oldsubparagraph\subparagraph
\renewcommand{\subparagraph}[1]{\oldsubparagraph{#1}\mbox{}}
\fi

%%% Use protect on footnotes to avoid problems with footnotes in titles
\let\rmarkdownfootnote\footnote%
\def\footnote{\protect\rmarkdownfootnote}

%%% Change title format to be more compact
\usepackage{titling}

% Create subtitle command for use in maketitle
\newcommand{\subtitle}[1]{
  \posttitle{
    \begin{center}\large#1\end{center}
    }
}

\setlength{\droptitle}{-2em}

  \title{Introduction to MoffittFunctions}
    \pretitle{\vspace{\droptitle}\centering\huge}
  \posttitle{\par}
    \author{William (Jimmy) Fulp}
    \preauthor{\centering\large\emph}
  \postauthor{\par}
      \predate{\centering\large\emph}
  \postdate{\par}
    \date{2019-01-29}

\hypersetup{colorlinks=true, linkcolor=blue}
\usepackage{booktabs}
\usepackage{longtable}
\usepackage{array}
\usepackage{multirow}
\usepackage{wrapfig}
\usepackage{colortbl}
\usepackage{float}
\usepackage{colortbl}
\usepackage{pdflscape}
\usepackage{tabu}
\usepackage{threeparttable}
\usepackage{threeparttablex}
\usepackage[normalem]{ulem}
\usepackage{makecell}

\begin{document}
\maketitle

{
\setcounter{tocdepth}{2}
\tableofcontents
}
\listoftables

\clearpage

\hypertarget{overview}{%
\section{Overview}\label{overview}}

MoffittFunctions is a collection of useful functions designed to assist
in analysis and creation of professional reports. The current
MoffittFunctions functions can be broken down to the following sections:

\begin{itemize}
\tightlist
\item
  Testing Functions

  \begin{itemize}
  \tightlist
  \item
    two\_samp\_bin\_test
  \item
    two\_samp\_cont\_test
  \item
    cor\_test
  \end{itemize}
\item
  Fancy Output Functions

  \begin{itemize}
  \tightlist
  \item
    pretty\_pvalues
  \item
    stat\_paste
  \item
    paste\_tbl\_grp
  \item
    pretty\_model\_output
  \item
    run\_pretty\_model\_output
  \item
    pretty\_km\_output
  \item
    run\_pretty\_km\_output
  \end{itemize}
\item
  Utility Functions

  \begin{itemize}
  \tightlist
  \item
    round\_away\_0
  \item
    get\_session\_info
  \item
    get\_full\_name
  \end{itemize}
\item
  Example Dataset

  \begin{itemize}
  \tightlist
  \item
    Bladder\_Cancer
  \end{itemize}
\end{itemize}

\hypertarget{getting-started}{%
\subsection{Getting Started}\label{getting-started}}

\begin{Shaded}
\begin{Highlighting}[]
\CommentTok{# Loading MoffittFunctions and Example Dataset}
\KeywordTok{library}\NormalTok{(MoffittFunctions)}
\KeywordTok{data}\NormalTok{(}\StringTok{"Bladder_Cancer"}\NormalTok{)}

\CommentTok{# Loading dplyr for this vignette code}
\KeywordTok{library}\NormalTok{(dplyr)}
\end{Highlighting}
\end{Shaded}

\hypertarget{moffitttemplates-package}{%
\subsection{MoffittTemplates Package}\label{moffitttemplates-package}}

The
\href{https://gitlab.moffitt.usf.edu:8000/ReproducibleResearch/R_Markdown_Templates}{MoffittTemplates}
package makes extensive use of the \texttt{MoffittFunctions} package,
and is a great way get started making professional statistical reports.

Code to initially download MoffittTemplates package:

\begin{Shaded}
\begin{Highlighting}[]
\NormalTok{cred =}\StringTok{ }\NormalTok{git2r}\OperatorTok{::}\KeywordTok{cred_ssh_key}\NormalTok{(}
    \DataTypeTok{publickey =} \StringTok{"MYPATH/.ssh/id_rsa.pub"}\NormalTok{, }
    \DataTypeTok{privatekey =} \StringTok{"MYPATH/.ssh/id_rsa"}\NormalTok{)}

\NormalTok{devtools}\OperatorTok{::}\KeywordTok{install_git}\NormalTok{(}
  \StringTok{"git@gitlab.moffitt.usf.edu:ReproducibleResearch/R_Markdown_Templates.git"}\NormalTok{, }
  \DataTypeTok{credentials =}\NormalTok{ cred)}
\end{Highlighting}
\end{Shaded}

RStudio you can \textbf{Global Options -\textgreater{} Git/SVN} to see
SSH path, and to make SSH key if needed

Once installed, in RStudio go to \textbf{File -\textgreater{} New File
-\textgreater{} R Markdown -\textgreater{} From Template -\textgreater{}
Moffitt PDF Report} to start a new Markdown report using the template.
Within the template there is code to load and make use of most of the
\texttt{MoffittFunctions} functionality.

\hypertarget{example-dataset}{%
\subsection{Example Dataset}\label{example-dataset}}

The \texttt{Bladder\_Cancer} dataset is a real world example dataset
used throughout this vignette and most example in the documentation. The
dataset is cleaned, using factor variables for categorical variables,
and also using labels for all variables (created by the
\texttt{Hmisc::label()} function).

\hypertarget{testing-functions}{%
\section{Testing Functions}\label{testing-functions}}

There are currently three testing functions, performing the appropriate
statistical test depending on the data and options, returning a p value.

\hypertarget{comparing-two-groups-binary-variable-for-a-binary-variable}{%
\subsection{Comparing Two Groups (Binary Variable) for a Binary
Variable}\label{comparing-two-groups-binary-variable-for-a-binary-variable}}

\texttt{two\_samp\_bin\_test()} is used for comparing a binary variable
to a binary (two group) variable, with options for Barnard, Fisher's
Exact, Chi-Sq, and McNemar tests.

\begin{Shaded}
\begin{Highlighting}[]

\KeywordTok{table}\NormalTok{(Bladder_Cancer}\OperatorTok{$}\NormalTok{Vital_Status, Bladder_Cancer}\OperatorTok{$}\NormalTok{PT0N0)}
\CommentTok{#>        }
\CommentTok{#>         No Completed Response Complete Response}
\CommentTok{#>   Alive                    74                33}
\CommentTok{#>   Dead                     57                 2}

\KeywordTok{two_samp_bin_test}\NormalTok{(}\DataTypeTok{x =}\NormalTok{ Bladder_Cancer}\OperatorTok{$}\NormalTok{Vital_Status, }\DataTypeTok{y =}\NormalTok{ Bladder_Cancer}\OperatorTok{$}\NormalTok{PT0N0, }
                  \DataTypeTok{method =} \StringTok{'fisher'}\NormalTok{)}
\CommentTok{#> [1] 0.00001536125}
\end{Highlighting}
\end{Shaded}

\hypertarget{comparing-two-groups-binary-variable-for-a-continuous-variable}{%
\subsection{Comparing Two Groups (Binary Variable) for a Continuous
Variable}\label{comparing-two-groups-binary-variable-for-a-continuous-variable}}

\texttt{two\_samp\_cont\_test()} is used for comparing a continuous
variable to a binary (two group) variable, with parametric (t.test) and
non-parametric (Wilcox Rank-Sum) options. Also pair data is allowed,
where there are parametric (paired t.test) and non-parametric (Wilcox
Signed-Rank) options.

\begin{Shaded}
\begin{Highlighting}[]

\KeywordTok{by}\NormalTok{(Bladder_Cancer}\OperatorTok{$}\NormalTok{Survival_Days, Bladder_Cancer}\OperatorTok{$}\NormalTok{PT0N0, summary)}
\CommentTok{#> Bladder_Cancer$PT0N0: No Completed Response}
\CommentTok{#>    Min. 1st Qu.  Median    Mean 3rd Qu.    Max. }
\CommentTok{#>     3.0   315.0   593.0   896.6  1292.5  3873.0 }
\CommentTok{#> -------------------------------------------------------- }
\CommentTok{#> Bladder_Cancer$PT0N0: Complete Response}
\CommentTok{#>    Min. 1st Qu.  Median    Mean 3rd Qu.    Max. }
\CommentTok{#>      34     538     817    1096    1581    3137}

\KeywordTok{two_samp_cont_test}\NormalTok{(}\DataTypeTok{x =}\NormalTok{ Bladder_Cancer}\OperatorTok{$}\NormalTok{Survival_Days, }\DataTypeTok{y =}\NormalTok{ Bladder_Cancer}\OperatorTok{$}\NormalTok{PT0N0, }
                   \DataTypeTok{method =} \StringTok{'wilcox'}\NormalTok{)}
\CommentTok{#> [1] 0.06303205}
\end{Highlighting}
\end{Shaded}

\hypertarget{comparing-two-continuous-variables-correlation}{%
\subsection{Comparing Two Continuous Variables
(Correlation)}\label{comparing-two-continuous-variables-correlation}}

\texttt{cor\_test()} is used for comparing two continuous variables,
with Pearson, Kendall, and Spearman methods. If Spearman method is
chosen and either variable has a tie the approximate distribution is use
in the \texttt{coin::spreaman\_test()} function. This is usually the
preferred method over the asymptotic approximation, which is the method
\texttt{stats:cor.test()} uses in cases of ties.

\begin{Shaded}
\begin{Highlighting}[]

\KeywordTok{cor}\NormalTok{(Bladder_Cancer}\OperatorTok{$}\NormalTok{Age_At_Diagnosis, Bladder_Cancer}\OperatorTok{$}\NormalTok{Survival_Days, }
    \DataTypeTok{method =} \StringTok{'spearman'}\NormalTok{)}
\CommentTok{#> [1] -0.13349}

\KeywordTok{cor_test}\NormalTok{(}\DataTypeTok{x =}\NormalTok{ Bladder_Cancer}\OperatorTok{$}\NormalTok{Age_At_Diagnosis, }\DataTypeTok{y =}\NormalTok{ Bladder_Cancer}\OperatorTok{$}\NormalTok{Survival_Days,}
         \DataTypeTok{method =} \StringTok{'spearman'}\NormalTok{)}
\CommentTok{#> [1] 0.0852}
\end{Highlighting}
\end{Shaded}

\hypertarget{fancy-output-functions}{%
\section{Fancy Output Functions}\label{fancy-output-functions}}

There are currently seven functions designed to produce professional
output that can easily printed in reports.

\hypertarget{p-values}{%
\subsection{P Values}\label{p-values}}

\texttt{pretty\_pvalues()} can be used on p values, rounding them to a
specified digit amount and using \textless{} for low p values, as
opposed to scientific notation (i.e.~``p \textless{} 0.0001'' if
rounding to 4 digits), allows options for emphasizing p-values and
specific characters for missing.

\begin{Shaded}
\begin{Highlighting}[]
\NormalTok{pvalue_example =}\StringTok{ }\KeywordTok{c}\NormalTok{(}\DecValTok{1}\NormalTok{, }\FloatTok{0.06753}\NormalTok{, }\FloatTok{0.004435}\NormalTok{, }\OtherTok{NA}\NormalTok{, }\FloatTok{1e-16}\NormalTok{, }\FloatTok{0.563533}\NormalTok{)}
\CommentTok{# For simple p value display}
\KeywordTok{pretty_pvalues}\NormalTok{(pvalue_example, }\DataTypeTok{digits =} \DecValTok{3}\NormalTok{)}
\CommentTok{#> [1] "1.000"  "0.068"  "0.004"  "---"    "<0.001" "0.564"}

\CommentTok{# For display in report}
\NormalTok{table_p_Values <-}\StringTok{ }\KeywordTok{pretty_pvalues}\NormalTok{(pvalue_example, }\DataTypeTok{digits =} \DecValTok{3}\NormalTok{, }\DataTypeTok{background =} \StringTok{"yellow"}\NormalTok{)}
\NormalTok{kableExtra}\OperatorTok{::}\KeywordTok{kable}\NormalTok{(table_p_Values, }\DataTypeTok{format =} \StringTok{'latex'}\NormalTok{, }\DataTypeTok{escape =} \OtherTok{FALSE}\NormalTok{, }
                  \DataTypeTok{col.names =} \KeywordTok{c}\NormalTok{(}\StringTok{"P-values"}\NormalTok{), }\DataTypeTok{caption =} \StringTok{'Fancy P Values'}\NormalTok{)}
\end{Highlighting}
\end{Shaded}

\begin{table}[t]

\caption{\label{tab:unnamed-chunk-6}Fancy P Values}
\centering
\begin{tabular}{l}
\hline
P-values\\
\hline
1.000\\
\hline
0.068\\
\hline
\cellcolor{yellow}{0.004}\\
\hline
---\\
\hline
\cellcolor{yellow}{<0.001}\\
\hline
0.564\\
\hline
\end{tabular}
\end{table}

You can also specify if you want \texttt{p=} pasted on the front of the
p values.

\hypertarget{basic-combining-of-variables}{%
\subsection{Basic Combining of
Variables}\label{basic-combining-of-variables}}

\texttt{stat\_paste()} is used to combine two or three statistics
together, allowing for different rounding and bound character
specifications. Common uses for this function are for:

\begin{itemize}
\tightlist
\item
  Mean (sd)
\item
  Median {[}min, max{]}
\item
  Estimate (SE of Estimate)
\item
  Estimate (95\% CI Lower Bound, Upper Bound)
\item
  Estimate/Statistic (p value)
\end{itemize}

\begin{Shaded}
\begin{Highlighting}[]
\CommentTok{# Simple Examples}
\KeywordTok{stat_paste}\NormalTok{(}\DataTypeTok{stat1 =} \FloatTok{2.45}\NormalTok{, }\DataTypeTok{stat2 =} \FloatTok{0.214}\NormalTok{, }\DataTypeTok{stat3 =} \FloatTok{55.3}\NormalTok{, }
           \DataTypeTok{digits =} \DecValTok{2}\NormalTok{, }\DataTypeTok{bound_char =} \StringTok{'['}\NormalTok{)}
\CommentTok{#> [1] "2.45 [0.21, 55.30]"}
\KeywordTok{stat_paste}\NormalTok{(}\DataTypeTok{stat1 =} \FloatTok{6.4864}\NormalTok{, }\DataTypeTok{stat2 =} \KeywordTok{pretty_pvalues}\NormalTok{(}\FloatTok{0.0004}\NormalTok{, }\DataTypeTok{digits =} \DecValTok{3}\NormalTok{), }
           \DataTypeTok{digits =} \DecValTok{3}\NormalTok{, }\DataTypeTok{bound_char =} \StringTok{'('}\NormalTok{)}
\CommentTok{#> [1] "6.486 (<0.001)"}


\NormalTok{Bladder_Cancer }\OperatorTok\StringTok{ }
\StringTok{  }\KeywordTok{group_by}\NormalTok{(Gender) }\OperatorTok\StringTok{ }
\StringTok{  }\KeywordTok{summarise}\NormalTok{(}\StringTok{`}\DataTypeTok{Survival Info (Median [Range])}\StringTok{`}\NormalTok{ =}\StringTok{ }
\StringTok{              }\KeywordTok{stat_paste}\NormalTok{(}\DataTypeTok{stat1 =} \KeywordTok{median}\NormalTok{(Survival_Months), }
                         \DataTypeTok{stat2 =} \KeywordTok{min}\NormalTok{(Survival_Months),}
                         \DataTypeTok{stat3 =} \KeywordTok{max}\NormalTok{(Survival_Months),}
                         \DataTypeTok{digits =} \DecValTok{2}\NormalTok{, }\DataTypeTok{bound_char =} \StringTok{'['}\NormalTok{))}
\end{Highlighting}
\end{Shaded}

\begin{tabular}{l|l}
\hline
Gender & Survival Info (Median [Range])\\
\hline
Female & 30.21 [1.90, 109.00]\\
\hline
Male & 20.40 [0.10, 127.19]\\
\hline
\end{tabular}

\begin{Shaded}
\begin{Highlighting}[]

\NormalTok{Bladder_Cancer }\OperatorTok\StringTok{ }
\StringTok{  }\KeywordTok{summarise}\NormalTok{(}
    \StringTok{`}\DataTypeTok{Age vs. Survival Months Cor(p value)}\StringTok{`}\NormalTok{ =}\StringTok{ }
\StringTok{      }\KeywordTok{stat_paste}\NormalTok{(}\DataTypeTok{stat1 =} \KeywordTok{cor}\NormalTok{(Age_At_Diagnosis, Survival_Months, }\DataTypeTok{method =} \StringTok{'spearman'}\NormalTok{), }
                 \DataTypeTok{stat2 =} 
                   \KeywordTok{pretty_pvalues}\NormalTok{(}
                     \KeywordTok{cor_test}\NormalTok{(Age_At_Diagnosis, Survival_Months, }\DataTypeTok{method =} \StringTok{'spearman'}\NormalTok{), }
                     \DataTypeTok{digits =} \DecValTok{3}\NormalTok{, }\DataTypeTok{include_p =} \OtherTok{TRUE}\NormalTok{),}
                 \DataTypeTok{digits =} \DecValTok{2}\NormalTok{, }\DataTypeTok{bound_char =} \StringTok{'('}\NormalTok{))}
\end{Highlighting}
\end{Shaded}

\begin{tabular}{l}
\hline
Age vs. Survival Months Cor(p value)\\
\hline
-0.13 (p=0.085)\\
\hline
\end{tabular}

\clearpage

\hypertarget{advanced-combining-of-variables}{%
\subsection{Advanced Combining of
Variables}\label{advanced-combining-of-variables}}

\texttt{paste\_tbl\_grp()} paste together information, often statistics,
from two groups. There are two predefined combinations: mean(sd) and
median{[}min,max{]}, but user may also paste any single measure
together.

\begin{Shaded}
\begin{Highlighting}[]

\NormalTok{summary_info <-}\StringTok{ }\NormalTok{Bladder_Cancer }\OperatorTok
\StringTok{ }\KeywordTok{group_by}\NormalTok{(Gender, Any_Downstaging) }\OperatorTok
\StringTok{ }\KeywordTok{summarise_at}\NormalTok{(}\StringTok{"Survival_Months"}\NormalTok{, }\KeywordTok{funs}\NormalTok{(}\DataTypeTok{n =}\NormalTok{ length, mean, sd, median, min, max)) }\OperatorTok
\StringTok{ }\NormalTok{tidyr}\OperatorTok{::}\KeywordTok{gather}\NormalTok{(variable, value, }\OperatorTok{-}\NormalTok{Any_Downstaging, }\OperatorTok{-}\NormalTok{Gender) }\OperatorTok
\StringTok{ }\NormalTok{tidyr}\OperatorTok{::}\KeywordTok{unite}\NormalTok{(var, Any_Downstaging, variable) }\OperatorTok\StringTok{ }
\StringTok{ }\NormalTok{tidyr}\OperatorTok{::}\KeywordTok{spread}\NormalTok{(var, value) }\OperatorTok
\StringTok{ }\KeywordTok{mutate}\NormalTok{(}\StringTok{`}\DataTypeTok{No Downstaging}\StringTok{`}\NormalTok{ =}\StringTok{ "No Downstaging"}\NormalTok{, }\DataTypeTok{Downstaging =} \StringTok{"Downstaging"}\NormalTok{) }\OperatorTok\StringTok{ }
\StringTok{ }\KeywordTok{paste_tbl_grp}\NormalTok{(}\DataTypeTok{vars_to_paste =} \KeywordTok{c}\NormalTok{(}\StringTok{'n'}\NormalTok{, }\StringTok{'mean_sd'}\NormalTok{, }\StringTok{'median_min_max'}\NormalTok{), }
               \DataTypeTok{first_name =} \StringTok{'No Downstaging'}\NormalTok{, }\DataTypeTok{second_name =} \StringTok{'Downstaging'}\NormalTok{)}

\NormalTok{kableExtra}\OperatorTok{::}\KeywordTok{kable}\NormalTok{(summary_info, }\DataTypeTok{format =} \StringTok{'latex'}\NormalTok{, }\DataTypeTok{escape =} \OtherTok{TRUE}\NormalTok{, }\DataTypeTok{booktabs =} \OtherTok{TRUE}\NormalTok{, }
                  \DataTypeTok{caption =} \StringTok{'Summary Information Comparison'}\NormalTok{) }\OperatorTok\StringTok{ }
\StringTok{  }\NormalTok{kableExtra}\OperatorTok{::}\KeywordTok{kable_styling}\NormalTok{(}\DataTypeTok{font_size =} \FloatTok{6.5}\NormalTok{) }\OperatorTok\StringTok{ }
\StringTok{  }\NormalTok{kableExtra}\OperatorTok{::}\KeywordTok{footnote}\NormalTok{(}
    \StringTok{'Summary Information for Downstaging vs. No-Downstaging, by Gender'}\NormalTok{)}
\end{Highlighting}
\end{Shaded}

\begin{table}[t]

\caption{\label{tab:unnamed-chunk-8}Summary Information Comparison}
\centering
\fontsize{6.5}{8.5}\selectfont
\begin{tabular}{lllll}
\toprule
Gender & Comparison & n\_comparison & mean\_sd\_comparison & median\_min\_max\_comparison\\
\midrule
Female & No Downstaging vs. Downstaging & 23 vs. 19 & 24 (19) vs. 40 (28) & 15 [2, 76] vs. 36 [7, 109]\\
Male & No Downstaging vs. Downstaging & 74 vs. 50 & 28 (27) vs. 34 (29) & 18 [1, 127] vs. 24 [0, 109]\\
\bottomrule
\multicolumn{5}{l}{\textit{Note: }}\\
\multicolumn{5}{l}{Summary Information for Downstaging vs. No-Downstaging, by Gender}\\
\end{tabular}
\end{table}

\hypertarget{model-output-functions}{%
\subsection{Model Output Functions}\label{model-output-functions}}

\texttt{pretty\_model\_output()} and
\texttt{run\_pretty\_model\_output()} are used to produce professional
tables for single or multiple Linear, Logistic, or Cox
Proportional-Hazards Regression Models, calculating estimates, odds
ratios, or hazard ratios, respectively, with confidence intervals. P
values are also produced. For categorical variables with 3+ levels
overall Type 3 p values are calculated (matches SAS's default overall p
values), in addition to p values comparing to the first level
(reference).

\texttt{pretty\_model\_output()} uses the model fits, while
\texttt{run\_pretty\_model\_output()} uses the variables and dataset,
running the desired model. The \texttt{run\_pretty\_model\_output()}
will use variable labels if they exist (created by the
\texttt{Hmisc::label()} function). Many details can be adjusted, such as
overall test method (``Wald'' or ``LR''), title (will be added as
column), confidence level, estimate and p value rounded digits,
significant alpha level for highlighting along with color, italic, and
bolding p value options, and latex or non-latex desired output.

In \texttt{run\_pretty\_model\_output()}, \texttt{y\_in},
\texttt{event\_in}, and \texttt{event\_level} are used defined
differently, depending on the type of model. For Linear Regression
\texttt{y\_in} is the dependent variable, and \texttt{event\_in} and
\texttt{event\_level} are left NULL. For Logistic Regression
\texttt{y\_in} is the dependent variable, \texttt{event\_level} is the
event level of the variable (i.e.~``1'' or ``Response''), and
\texttt{event\_in} is left NULL. For Cox Regression \texttt{y\_in} is
the time component, \texttt{event\_in} is the event status variable, and
\texttt{event\_level} is the event level of the \texttt{event\_in}
variable (i.e.~``1'' or ``Dead'').

\clearpage

\hypertarget{linear-regression-example}{%
\subsubsection{Linear Regression
Example}\label{linear-regression-example}}

\begin{Shaded}
\begin{Highlighting}[]
\CommentTok{# Using pretty_model_output() for a single multivariable model}

\NormalTok{my_fit <-}\StringTok{ }\KeywordTok{lm}\NormalTok{(}
\NormalTok{  Surgery_Year }\OperatorTok{~}\StringTok{ }\NormalTok{Age_At_Diagnosis }\OperatorTok{+}\StringTok{ }\NormalTok{Gender }\OperatorTok{+}\StringTok{ }\NormalTok{Clinical_Stage_Grouped }\OperatorTok{+}\StringTok{ }
\StringTok{    }\NormalTok{Histology_Grouped, }\DataTypeTok{data =}\NormalTok{ Bladder_Cancer)}
\KeywordTok{pretty_model_output}\NormalTok{(}\DataTypeTok{fit =}\NormalTok{ my_fit, }\DataTypeTok{model_data =}\NormalTok{ Bladder_Cancer)}
\end{Highlighting}
\end{Shaded}

\begin{tabular}{l|l|l|l|l}
\hline
Variable & Level & Est (95\% CI) & P Value & Overall P Value\\
\hline
Age &  & 0.034 (-0.014, 0.081) & 0.1617 & \\
\hline
Gender & Female & 1.0 (Reference) & - & \\
\hline
Gender & Male & -0.414 (-1.379, 0.551) & 0.3983 & \\
\hline
Clinical AJCC Stage & Stage I/II (<=T2NxMx) & 1.0 (Reference) & - & 0.0004\\
\hline
Clinical AJCC Stage & Stage III (T3NxMx) & -2.190 (-3.259, -1.120) & <0.0001 & \\
\hline
Clinical AJCC Stage & Stage IV (T4NxMx) & -1.007 (-2.490, 0.476) & 0.1817 & \\
\hline
Histology & Pure Urothelial Neoplasia & 1.0 (Reference) & - & 0.0038\\
\hline
Histology & Mixed Tumors & 1.109 (0.093, 2.126) & 0.0326 & \\
\hline
Histology & Variant Histology & 1.661 (0.605, 2.717) & 0.0022 & \\
\hline
\end{tabular}

\begin{Shaded}
\begin{Highlighting}[]

\NormalTok{vars_to_run =}\StringTok{ }\KeywordTok{c}\NormalTok{(}\StringTok{'Age_At_Diagnosis'}\NormalTok{, }\StringTok{'Gender'}\NormalTok{, }\StringTok{'Clinical_Stage_Grouped'}\NormalTok{, }\StringTok{'Histology_Grouped'}\NormalTok{)}

\CommentTok{# Using run_pretty_model_output() for multiple univariate linear regression models}
\CommentTok{# Use purrr::map_dfr function to run the run_pretty_model_output() multiple times}
\NormalTok{linear_univariate_output <-}\StringTok{ }\NormalTok{purrr}\OperatorTok{::}\KeywordTok{map_dfr}\NormalTok{(}
\NormalTok{  vars_to_run, run_pretty_model_output, }\DataTypeTok{model_data =}\NormalTok{ Bladder_Cancer, }
  \DataTypeTok{y_in =} \StringTok{'Surgery_Year'}\NormalTok{, }\DataTypeTok{event_in =} \OtherTok{NULL}\NormalTok{, }\DataTypeTok{event_level =} \OtherTok{NULL}\NormalTok{, }
  \DataTypeTok{output_type =} \StringTok{'latex'}\NormalTok{) }

\CommentTok{#Use kableExtra to make fancy output}
\NormalTok{kableExtra}\OperatorTok{::}\KeywordTok{kable}\NormalTok{(}
\NormalTok{  linear_univariate_output, }\StringTok{'latex'}\NormalTok{, }\DataTypeTok{escape =}\NormalTok{ F, }\DataTypeTok{booktabs =} \OtherTok{TRUE}\NormalTok{, }
  \DataTypeTok{linesep =} \StringTok{''}\NormalTok{, }\DataTypeTok{caption =} \StringTok{'Linear Regression Univariate Models'}\NormalTok{) }\OperatorTok
\StringTok{  }\NormalTok{kableExtra}\OperatorTok{::}\KeywordTok{collapse_rows}\NormalTok{(}\KeywordTok{c}\NormalTok{(}\DecValTok{1}\OperatorTok{:}\DecValTok{2}\NormalTok{), }\DataTypeTok{row_group_label_position =} \StringTok{'stack'}\NormalTok{,}
                            \DataTypeTok{headers_to_remove =} \DecValTok{1}\OperatorTok{:}\DecValTok{2}\NormalTok{, }\DataTypeTok{latex_hline =} \StringTok{'major'}\NormalTok{) }\OperatorTok\StringTok{ }
\StringTok{  }\NormalTok{kableExtra}\OperatorTok{::}\KeywordTok{kable_styling}\NormalTok{(}\DataTypeTok{font_size =} \DecValTok{10}\NormalTok{)}
\end{Highlighting}
\end{Shaded}

\begin{table}[t]

\caption{\label{tab:unnamed-chunk-10}Linear Regression Univariate Models}
\centering
\fontsize{10}{12}\selectfont
\begin{tabular}{llllll}
\toprule
 &  & Est (95\% CI) & P Value & Overall P Value & n\\
\midrule
\addlinespace[0.3em]
\multicolumn{6}{l}{\textbf{Age}}\\
\hspace{1em} &  & 0.033 (-0.018, 0.083) & 0.2040 &  & n=166\\
\cmidrule{1-6}
\addlinespace[0.3em]
\multicolumn{6}{l}{\textbf{Gender}}\\
\hspace{1em} & Female & 1.0 (Reference) & --- &  & n=166\\

\hspace{1em} & Male & -0.170 (-1.176, 0.836) & 0.7388 &  & \\
\cmidrule{1-6}
\addlinespace[0.3em]
\multicolumn{6}{l}{\textbf{Clinical AJCC Stage}}\\
\hspace{1em} & Stage I/II (<=T2NxMx) & 1.0 (Reference) & --- & \cellcolor{yellow}{0.0008} & n=166\\

\hspace{1em} & Stage III (T3NxMx) & -2.109 (-3.187, -1.031) & \cellcolor{yellow}{0.0002} &  & \\

\hspace{1em} & Stage IV (T4NxMx) & -0.672 (-2.113, 0.770) & 0.3588 &  & \\
\cmidrule{1-6}
\addlinespace[0.3em]
\multicolumn{6}{l}{\textbf{Histology}}\\
\hspace{1em} & Pure Urothelial Neoplasia & 1.0 (Reference) & --- & \cellcolor{yellow}{0.0033} & n=166\\

\hspace{1em} & Mixed Tumors & 0.831 (-0.185, 1.846) & 0.1083 &  & \\

\hspace{1em} & Variant Histology & 1.870 (0.779, 2.961) & \cellcolor{yellow}{0.0009} &  & \\
\bottomrule
\end{tabular}
\end{table}

\begin{Shaded}
\begin{Highlighting}[]

\CommentTok{# Using run_pretty_model_output() for a multivariable linear regression model}
\NormalTok{linear_multivariable_output <-}\StringTok{ }\KeywordTok{run_pretty_model_output}\NormalTok{(}
  \DataTypeTok{x_in =}\NormalTok{ vars_to_run, }\DataTypeTok{model_data =}\NormalTok{ Bladder_Cancer, }\DataTypeTok{y_in =} \StringTok{'Surgery_Year'}\NormalTok{, }
  \DataTypeTok{event_in =} \OtherTok{NULL}\NormalTok{, }\DataTypeTok{event_level =} \OtherTok{NULL}\NormalTok{, }\DataTypeTok{output_type =} \StringTok{'latex'}\NormalTok{)}

\CommentTok{#Use kableExtra to make fancy output}
\NormalTok{kableExtra}\OperatorTok{::}\KeywordTok{kable}\NormalTok{(}
\NormalTok{  linear_multivariable_output }\OperatorTok\StringTok{ }\NormalTok{dplyr}\OperatorTok{::}\KeywordTok{select}\NormalTok{(}\OperatorTok{-}\NormalTok{n), }\StringTok{'latex'}\NormalTok{, }
  \DataTypeTok{escape =}\NormalTok{ F, }\DataTypeTok{booktabs =} \OtherTok{TRUE}\NormalTok{, }\DataTypeTok{linesep =} \StringTok{''}\NormalTok{, }
  \DataTypeTok{caption =} \StringTok{'Linear Regression Multivariable Model'}\NormalTok{) }\OperatorTok
\StringTok{  }\NormalTok{kableExtra}\OperatorTok{::}\KeywordTok{collapse_rows}\NormalTok{(}\KeywordTok{c}\NormalTok{(}\DecValTok{1}\OperatorTok{:}\DecValTok{2}\NormalTok{), }\DataTypeTok{row_group_label_position =} \StringTok{'stack'}\NormalTok{,}
                            \DataTypeTok{headers_to_remove =} \DecValTok{1}\OperatorTok{:}\DecValTok{2}\NormalTok{, }\DataTypeTok{latex_hline =} \StringTok{'major'}\NormalTok{) }\OperatorTok\StringTok{ }
\StringTok{  }\NormalTok{kableExtra}\OperatorTok{::}\KeywordTok{kable_styling}\NormalTok{(}\DataTypeTok{font_size =} \DecValTok{10}\NormalTok{) }\OperatorTok\StringTok{ }
\StringTok{  }\NormalTok{kableExtra}\OperatorTok{::}\KeywordTok{footnote}\NormalTok{(}
    \KeywordTok{paste0}\NormalTok{(}\StringTok{'Model sample size is '}\NormalTok{, }
\NormalTok{           linear_multivariable_output }\OperatorTok\StringTok{ }\KeywordTok{select}\NormalTok{(n) }\OperatorTok\StringTok{ }\KeywordTok{slice}\NormalTok{(}\DecValTok{1}\NormalTok{)))}
\end{Highlighting}
\end{Shaded}

\begin{table}[t]

\caption{\label{tab:unnamed-chunk-11}Linear Regression Multivariable Model}
\centering
\fontsize{10}{12}\selectfont
\begin{tabular}{lllll}
\toprule
 &  & Est (95\% CI) & P Value & Overall P Value\\
\midrule
\addlinespace[0.3em]
\multicolumn{5}{l}{\textbf{Age}}\\
\hspace{1em} &  & 0.034 (-0.014, 0.081) & 0.1617 & \\
\cmidrule{1-5}
\addlinespace[0.3em]
\multicolumn{5}{l}{\textbf{Gender}}\\
\hspace{1em} & Female & 1.0 (Reference) & --- & \\

\hspace{1em} & Male & -0.414 (-1.379, 0.551) & 0.3983 & \\
\cmidrule{1-5}
\addlinespace[0.3em]
\multicolumn{5}{l}{\textbf{Clinical AJCC Stage}}\\
\hspace{1em} & Stage I/II (<=T2NxMx) & 1.0 (Reference) & --- & \cellcolor{yellow}{0.0004}\\

\hspace{1em} & Stage III (T3NxMx) & -2.190 (-3.259, -1.120) & \cellcolor{yellow}{<0.0001} & \\

\hspace{1em} & Stage IV (T4NxMx) & -1.007 (-2.490, 0.476) & 0.1817 & \\
\cmidrule{1-5}
\addlinespace[0.3em]
\multicolumn{5}{l}{\textbf{Histology}}\\
\hspace{1em} & Pure Urothelial Neoplasia & 1.0 (Reference) & --- & \cellcolor{yellow}{0.0038}\\

\hspace{1em} & Mixed Tumors & 1.109 (0.093, 2.126) & \cellcolor{yellow}{0.0326} & \\

\hspace{1em} & Variant Histology & 1.661 (0.605, 2.717) & \cellcolor{yellow}{0.0022} & \\
\bottomrule
\multicolumn{5}{l}{\textit{Note: }}\\
\multicolumn{5}{l}{Model sample size is n=166}\\
\end{tabular}
\end{table}

\clearpage

\hypertarget{logistic-regression-example}{%
\subsubsection{Logistic Regression
Example}\label{logistic-regression-example}}

\begin{Shaded}
\begin{Highlighting}[]
\CommentTok{# Using pretty_model_output() for a single multivariable model}

\NormalTok{my_fit <-}\StringTok{ }\KeywordTok{glm}\NormalTok{(}
\NormalTok{  Any_Downstaging }\OperatorTok{==}\StringTok{ 'Downstaging'} \OperatorTok{~}\StringTok{ }\NormalTok{Age_At_Diagnosis }\OperatorTok{+}\StringTok{ }\NormalTok{Gender }\OperatorTok{+}\StringTok{ }\NormalTok{Clinical_Stage_Grouped }\OperatorTok{+}\StringTok{ }
\StringTok{    }\NormalTok{Histology_Grouped, }\DataTypeTok{data =}\NormalTok{ Bladder_Cancer, }\DataTypeTok{family =} \KeywordTok{binomial}\NormalTok{(}\DataTypeTok{link =} \StringTok{"logit"}\NormalTok{))}
\KeywordTok{pretty_model_output}\NormalTok{(}\DataTypeTok{fit =}\NormalTok{ my_fit, }\DataTypeTok{model_data =}\NormalTok{ Bladder_Cancer)}
\end{Highlighting}
\end{Shaded}

\begin{tabular}{l|l|l|l|l}
\hline
Variable & Level & OR (95\% CI) & P Value & Overall P Value\\
\hline
Age &  & 1.027 (0.989, 1.069) & 0.1737 & \\
\hline
Gender & Female & 1.0 (Reference) & - & \\
\hline
Gender & Male & 0.612 (0.281, 1.318) & 0.2111 & \\
\hline
Clinical AJCC Stage & Stage I/II (<=T2NxMx) & 1.0 (Reference) & - & 0.0168\\
\hline
Clinical AJCC Stage & Stage III (T3NxMx) & 0.295 (0.111, 0.714) & 0.0095 & \\
\hline
Clinical AJCC Stage & Stage IV (T4NxMx) & 0.356 (0.099, 1.131) & 0.0912 & \\
\hline
Histology & Pure Urothelial Neoplasia & 1.0 (Reference) & - & 0.1587\\
\hline
Histology & Mixed Tumors & 1.102 (0.495, 2.464) & 0.8114 & \\
\hline
Histology & Variant Histology & 0.457 (0.188, 1.058) & 0.0740 & \\
\hline
\end{tabular}

\begin{Shaded}
\begin{Highlighting}[]

\NormalTok{vars_to_run =}\StringTok{ }\KeywordTok{c}\NormalTok{(}\StringTok{'Age_At_Diagnosis'}\NormalTok{, }\StringTok{'Gender'}\NormalTok{, }\StringTok{'Clinical_Stage_Grouped'}\NormalTok{, }\StringTok{'Histology_Grouped'}\NormalTok{)}

\CommentTok{# Using run_pretty_model_output() for multiple univariate logistic regression models}
\CommentTok{# Use purrr::map_dfr function to run the run_pretty_model_output() multiple times}
\NormalTok{logistic_univariate_output <-}\StringTok{ }\NormalTok{purrr}\OperatorTok{::}\KeywordTok{map_dfr}\NormalTok{(}
\NormalTok{  vars_to_run, run_pretty_model_output, }\DataTypeTok{model_data =}\NormalTok{ Bladder_Cancer, }
  \DataTypeTok{y_in =} \StringTok{'Any_Downstaging'}\NormalTok{, }\DataTypeTok{event_in =} \OtherTok{NULL}\NormalTok{, }\DataTypeTok{event_level =} \StringTok{'Downstaging'}\NormalTok{, }
  \DataTypeTok{output_type =} \StringTok{'latex'}\NormalTok{) }

\CommentTok{#Use kableExtra to make fancy output}
\NormalTok{kableExtra}\OperatorTok{::}\KeywordTok{kable}\NormalTok{(}
\NormalTok{  logistic_univariate_output, }\StringTok{'latex'}\NormalTok{, }\DataTypeTok{escape =}\NormalTok{ F, }\DataTypeTok{booktabs =} \OtherTok{TRUE}\NormalTok{, }
  \DataTypeTok{linesep =} \StringTok{''}\NormalTok{, }\DataTypeTok{caption =} \StringTok{'Logistic Regression Univariate Models'}\NormalTok{) }\OperatorTok
\StringTok{  }\NormalTok{kableExtra}\OperatorTok{::}\KeywordTok{collapse_rows}\NormalTok{(}\KeywordTok{c}\NormalTok{(}\DecValTok{1}\OperatorTok{:}\DecValTok{2}\NormalTok{), }\DataTypeTok{row_group_label_position =} \StringTok{'stack'}\NormalTok{,}
                            \DataTypeTok{headers_to_remove =} \DecValTok{1}\OperatorTok{:}\DecValTok{2}\NormalTok{, }\DataTypeTok{latex_hline =} \StringTok{'major'}\NormalTok{) }\OperatorTok\StringTok{ }
\StringTok{  }\NormalTok{kableExtra}\OperatorTok{::}\KeywordTok{kable_styling}\NormalTok{(}\DataTypeTok{font_size =} \DecValTok{10}\NormalTok{)}
\end{Highlighting}
\end{Shaded}

\begin{table}[t]

\caption{\label{tab:unnamed-chunk-13}Logistic Regression Univariate Models}
\centering
\fontsize{10}{12}\selectfont
\begin{tabular}{llllll}
\toprule
 &  & OR (95\% CI) & P Value & Overall P Value & n\\
\midrule
\addlinespace[0.3em]
\multicolumn{6}{l}{\textbf{Age}}\\
\hspace{1em} &  & 1.020 (0.984, 1.059) & 0.2785 &  & n=166\\
\cmidrule{1-6}
\addlinespace[0.3em]
\multicolumn{6}{l}{\textbf{Gender}}\\
\hspace{1em} & Female & 1.0 (Reference) & --- &  & n=166\\

\hspace{1em} & Male & 0.818 (0.404, 1.667) & 0.5767 &  & \\
\cmidrule{1-6}
\addlinespace[0.3em]
\multicolumn{6}{l}{\textbf{Clinical AJCC Stage}}\\
\hspace{1em} & Stage I/II (<=T2NxMx) & 1.0 (Reference) & --- & 0.0552 & n=166\\

\hspace{1em} & Stage III (T3NxMx) & 0.369 (0.145, 0.857) & \cellcolor{yellow}{0.0261} &  & \\

\hspace{1em} & Stage IV (T4NxMx) & 0.503 (0.151, 1.475) & 0.2283 &  & \\
\cmidrule{1-6}
\addlinespace[0.3em]
\multicolumn{6}{l}{\textbf{Histology}}\\
\hspace{1em} & Pure Urothelial Neoplasia & 1.0 (Reference) & --- & 0.3672 & n=166\\

\hspace{1em} & Mixed Tumors & 0.864 (0.410, 1.800) & 0.6974 &  & \\

\hspace{1em} & Variant Histology & 0.550 (0.233, 1.237) & 0.1570 &  & \\
\bottomrule
\end{tabular}
\end{table}

\begin{Shaded}
\begin{Highlighting}[]

\CommentTok{# Using run_pretty_model_output() for a multivariable logistic regression model}
\NormalTok{logistic_multivariable_output <-}\StringTok{ }\KeywordTok{run_pretty_model_output}\NormalTok{(}
  \DataTypeTok{x_in =}\NormalTok{ vars_to_run, }\DataTypeTok{model_data =}\NormalTok{ Bladder_Cancer, }\DataTypeTok{y_in =} \StringTok{'Any_Downstaging'}\NormalTok{, }
  \DataTypeTok{event_in =} \OtherTok{NULL}\NormalTok{, }\DataTypeTok{event_level =} \StringTok{'Downstaging'}\NormalTok{, }\DataTypeTok{output_type =} \StringTok{'latex'}\NormalTok{)}

\CommentTok{#Use kableExtra to make fancy output}
\NormalTok{kableExtra}\OperatorTok{::}\KeywordTok{kable}\NormalTok{(}
\NormalTok{  logistic_multivariable_output }\OperatorTok\StringTok{ }\NormalTok{dplyr}\OperatorTok{::}\KeywordTok{select}\NormalTok{(}\OperatorTok{-}\NormalTok{n), }\StringTok{'latex'}\NormalTok{, }
  \DataTypeTok{escape =}\NormalTok{ F, }\DataTypeTok{booktabs =} \OtherTok{TRUE}\NormalTok{, }\DataTypeTok{linesep =} \StringTok{''}\NormalTok{, }
  \DataTypeTok{caption =} \StringTok{'Logistic Regression Multivariable Model'}\NormalTok{) }\OperatorTok
\StringTok{  }\NormalTok{kableExtra}\OperatorTok{::}\KeywordTok{collapse_rows}\NormalTok{(}\KeywordTok{c}\NormalTok{(}\DecValTok{1}\OperatorTok{:}\DecValTok{2}\NormalTok{), }\DataTypeTok{row_group_label_position =} \StringTok{'stack'}\NormalTok{,}
                            \DataTypeTok{headers_to_remove =} \DecValTok{1}\OperatorTok{:}\DecValTok{2}\NormalTok{, }\DataTypeTok{latex_hline =} \StringTok{'major'}\NormalTok{) }\OperatorTok\StringTok{ }
\StringTok{  }\NormalTok{kableExtra}\OperatorTok{::}\KeywordTok{kable_styling}\NormalTok{(}\DataTypeTok{font_size =} \DecValTok{10}\NormalTok{) }\OperatorTok\StringTok{ }
\StringTok{  }\NormalTok{kableExtra}\OperatorTok{::}\KeywordTok{footnote}\NormalTok{(}
    \KeywordTok{paste0}\NormalTok{(}\StringTok{'Model sample size is '}\NormalTok{, }
\NormalTok{           logistic_multivariable_output }\OperatorTok\StringTok{ }\KeywordTok{select}\NormalTok{(n) }\OperatorTok\StringTok{ }\KeywordTok{slice}\NormalTok{(}\DecValTok{1}\NormalTok{)))}
\end{Highlighting}
\end{Shaded}

\begin{table}[t]

\caption{\label{tab:unnamed-chunk-14}Logistic Regression Multivariable Model}
\centering
\fontsize{10}{12}\selectfont
\begin{tabular}{lllll}
\toprule
 &  & OR (95\% CI) & P Value & Overall P Value\\
\midrule
\addlinespace[0.3em]
\multicolumn{5}{l}{\textbf{Age}}\\
\hspace{1em} &  & 1.027 (0.989, 1.069) & 0.1737 & \\
\cmidrule{1-5}
\addlinespace[0.3em]
\multicolumn{5}{l}{\textbf{Gender}}\\
\hspace{1em} & Female & 1.0 (Reference) & --- & \\

\hspace{1em} & Male & 0.612 (0.281, 1.318) & 0.2111 & \\
\cmidrule{1-5}
\addlinespace[0.3em]
\multicolumn{5}{l}{\textbf{Clinical AJCC Stage}}\\
\hspace{1em} & Stage I/II (<=T2NxMx) & 1.0 (Reference) & --- & \cellcolor{yellow}{0.0168}\\

\hspace{1em} & Stage III (T3NxMx) & 0.295 (0.111, 0.714) & \cellcolor{yellow}{0.0095} & \\

\hspace{1em} & Stage IV (T4NxMx) & 0.356 (0.099, 1.131) & 0.0912 & \\
\cmidrule{1-5}
\addlinespace[0.3em]
\multicolumn{5}{l}{\textbf{Histology}}\\
\hspace{1em} & Pure Urothelial Neoplasia & 1.0 (Reference) & --- & 0.1587\\

\hspace{1em} & Mixed Tumors & 1.102 (0.495, 2.464) & 0.8114 & \\

\hspace{1em} & Variant Histology & 0.457 (0.188, 1.058) & 0.0740 & \\
\bottomrule
\multicolumn{5}{l}{\textit{Note: }}\\
\multicolumn{5}{l}{Model sample size is n=166}\\
\end{tabular}
\end{table}

\clearpage

\hypertarget{cox-proportional-hazards-regression-example}{%
\subsubsection{Cox Proportional-Hazards Regression
Example}\label{cox-proportional-hazards-regression-example}}

\begin{Shaded}
\begin{Highlighting}[]
\CommentTok{# Using pretty_model_output() for a single multivariable model}

\NormalTok{surv_obj <-}\StringTok{ }\NormalTok{survival}\OperatorTok{::}\KeywordTok{Surv}\NormalTok{(Bladder_Cancer}\OperatorTok{$}\NormalTok{Survival_Months, Bladder_Cancer}\OperatorTok{$}\NormalTok{Vital_Status }\OperatorTok{==}\StringTok{ 'Dead'}\NormalTok{)   }
\NormalTok{my_fit <-}\StringTok{ }\NormalTok{survival}\OperatorTok{::}\KeywordTok{coxph}\NormalTok{(}
\NormalTok{  surv_obj }\OperatorTok{~}\StringTok{ }\NormalTok{Age_At_Diagnosis }\OperatorTok{+}\StringTok{ }\NormalTok{Gender }\OperatorTok{+}\StringTok{ }\NormalTok{Clinical_Stage_Grouped }\OperatorTok{+}\StringTok{ }
\StringTok{    }\NormalTok{Histology_Grouped, }\DataTypeTok{data =}\NormalTok{ Bladder_Cancer)}
\KeywordTok{pretty_model_output}\NormalTok{(}\DataTypeTok{fit =}\NormalTok{ my_fit, }\DataTypeTok{model_data =}\NormalTok{ Bladder_Cancer)}
\end{Highlighting}
\end{Shaded}

\begin{tabular}{l|l|l|l|l}
\hline
Variable & Level & HR (95\% CI) & P Value & Overall P Value\\
\hline
Age &  & 0.997 (0.967, 1.028) & 0.8384 & \\
\hline
Gender & Female & 1.0 (Reference) & - & \\
\hline
Gender & Male & 1.677 (0.863, 3.256) & 0.1269 & \\
\hline
Clinical AJCC Stage & Stage I/II (<=T2NxMx) & 1.0 (Reference) & - & 0.0349\\
\hline
Clinical AJCC Stage & Stage III (T3NxMx) & 2.217 (1.209, 4.065) & 0.0101 & \\
\hline
Clinical AJCC Stage & Stage IV (T4NxMx) & 1.683 (0.662, 4.275) & 0.2740 & \\
\hline
Histology & Pure Urothelial Neoplasia & 1.0 (Reference) & - & 0.5314\\
\hline
Histology & Mixed Tumors & 1.276 (0.678, 2.400) & 0.4501 & \\
\hline
Histology & Variant Histology & 1.419 (0.715, 2.814) & 0.3168 & \\
\hline
\end{tabular}

\begin{Shaded}
\begin{Highlighting}[]

\NormalTok{vars_to_run =}\StringTok{ }\KeywordTok{c}\NormalTok{(}\StringTok{'Age_At_Diagnosis'}\NormalTok{, }\StringTok{'Gender'}\NormalTok{, }\StringTok{'Clinical_Stage_Grouped'}\NormalTok{, }\StringTok{'Histology_Grouped'}\NormalTok{)}

\CommentTok{# Using run_pretty_model_output() for multiple univariate Cox regression models}
\CommentTok{# Use purrr::map_dfr function to run the run_pretty_model_output() multiple times}
\NormalTok{cox_univariate_output <-}\StringTok{ }\NormalTok{purrr}\OperatorTok{::}\KeywordTok{map_dfr}\NormalTok{(}
\NormalTok{  vars_to_run, run_pretty_model_output, }\DataTypeTok{model_data =}\NormalTok{ Bladder_Cancer, }
  \DataTypeTok{y_in =} \StringTok{'Survival_Months'}\NormalTok{, }\DataTypeTok{event_in =} \StringTok{'Vital_Status'}\NormalTok{, }\DataTypeTok{event_level =} \StringTok{'Dead'}\NormalTok{, }
  \DataTypeTok{output_type =} \StringTok{'latex'}\NormalTok{) }

\CommentTok{#Use kableExtra to make fancy output}
\NormalTok{kableExtra}\OperatorTok{::}\KeywordTok{kable}\NormalTok{(}
\NormalTok{  cox_univariate_output, }\StringTok{'latex'}\NormalTok{, }\DataTypeTok{escape =}\NormalTok{ F, }\DataTypeTok{booktabs =} \OtherTok{TRUE}\NormalTok{, }
  \DataTypeTok{linesep =} \StringTok{''}\NormalTok{, }\DataTypeTok{caption =} \StringTok{'Cox Proportional-Hazards Regression Univariate Models'}\NormalTok{) }\OperatorTok
\StringTok{  }\NormalTok{kableExtra}\OperatorTok{::}\KeywordTok{collapse_rows}\NormalTok{(}\KeywordTok{c}\NormalTok{(}\DecValTok{1}\OperatorTok{:}\DecValTok{2}\NormalTok{), }\DataTypeTok{row_group_label_position =} \StringTok{'stack'}\NormalTok{,}
                            \DataTypeTok{headers_to_remove =} \DecValTok{1}\OperatorTok{:}\DecValTok{2}\NormalTok{, }\DataTypeTok{latex_hline =} \StringTok{'major'}\NormalTok{) }\OperatorTok\StringTok{ }
\StringTok{  }\NormalTok{kableExtra}\OperatorTok{::}\KeywordTok{kable_styling}\NormalTok{(}\DataTypeTok{font_size =} \DecValTok{10}\NormalTok{)}
\end{Highlighting}
\end{Shaded}

\begin{table}[t]

\caption{\label{tab:unnamed-chunk-16}Cox Proportional-Hazards Regression Univariate Models}
\centering
\fontsize{10}{12}\selectfont
\begin{tabular}{llllll}
\toprule
 &  & HR (95\% CI) & P Value & Overall P Value & n (events)\\
\midrule
\addlinespace[0.3em]
\multicolumn{6}{l}{\textbf{Age}}\\
\hspace{1em} &  & 1.004 (0.973, 1.035) & 0.8063 &  & n=166 (59)\\
\cmidrule{1-6}
\addlinespace[0.3em]
\multicolumn{6}{l}{\textbf{Gender}}\\
\hspace{1em} & Female & 1.0 (Reference) & --- &  & n=166 (59)\\

\hspace{1em} & Male & 1.412 (0.748, 2.663) & 0.2872 &  & \\
\cmidrule{1-6}
\addlinespace[0.3em]
\multicolumn{6}{l}{\textbf{Clinical AJCC Stage}}\\
\hspace{1em} & Stage I/II (<=T2NxMx) & 1.0 (Reference) & --- & 0.0503 & n=166 (59)\\

\hspace{1em} & Stage III (T3NxMx) & 1.985 (1.128, 3.494) & \cellcolor{yellow}{0.0174} &  & \\

\hspace{1em} & Stage IV (T4NxMx) & 1.632 (0.684, 3.891) & 0.2693 &  & \\
\cmidrule{1-6}
\addlinespace[0.3em]
\multicolumn{6}{l}{\textbf{Histology}}\\
\hspace{1em} & Pure Urothelial Neoplasia & 1.0 (Reference) & --- & 0.3606 & n=166 (59)\\

\hspace{1em} & Mixed Tumors & 1.540 (0.846, 2.802) & 0.1578 &  & \\

\hspace{1em} & Variant Histology & 1.257 (0.640, 2.470) & 0.5062 &  & \\
\bottomrule
\end{tabular}
\end{table}

\begin{Shaded}
\begin{Highlighting}[]

\CommentTok{# Using run_pretty_model_output() for a multivariable Cox regression model}
\NormalTok{cox_multivariable_output <-}\StringTok{ }\KeywordTok{run_pretty_model_output}\NormalTok{(}
  \DataTypeTok{x_in =}\NormalTok{ vars_to_run, }\DataTypeTok{model_data =}\NormalTok{ Bladder_Cancer, }\DataTypeTok{y_in =} \StringTok{'Survival_Months'}\NormalTok{, }
  \DataTypeTok{event_in =} \StringTok{'Vital_Status'}\NormalTok{, }\DataTypeTok{event_level =} \StringTok{'Dead'}\NormalTok{, }\DataTypeTok{output_type =} \StringTok{'latex'}\NormalTok{)}

\CommentTok{#Use kableExtra to make fancy output}
\NormalTok{kableExtra}\OperatorTok{::}\KeywordTok{kable}\NormalTok{(}
\NormalTok{  cox_multivariable_output }\OperatorTok\StringTok{ }\NormalTok{dplyr}\OperatorTok{::}\KeywordTok{select}\NormalTok{(}\OperatorTok{-}\StringTok{`}\DataTypeTok{n (events)}\StringTok{`}\NormalTok{), }\StringTok{'latex'}\NormalTok{, }
  \DataTypeTok{escape =}\NormalTok{ F, }\DataTypeTok{booktabs =} \OtherTok{TRUE}\NormalTok{, }\DataTypeTok{linesep =} \StringTok{''}\NormalTok{, }
  \DataTypeTok{caption =} \StringTok{'Cox Proportional-Hazards Regression Multivariable Model'}\NormalTok{) }\OperatorTok
\StringTok{  }\NormalTok{kableExtra}\OperatorTok{::}\KeywordTok{collapse_rows}\NormalTok{(}\KeywordTok{c}\NormalTok{(}\DecValTok{1}\OperatorTok{:}\DecValTok{2}\NormalTok{), }\DataTypeTok{row_group_label_position =} \StringTok{'stack'}\NormalTok{,}
                            \DataTypeTok{headers_to_remove =} \DecValTok{1}\OperatorTok{:}\DecValTok{2}\NormalTok{, }\DataTypeTok{latex_hline =} \StringTok{'major'}\NormalTok{) }\OperatorTok\StringTok{ }
\StringTok{  }\NormalTok{kableExtra}\OperatorTok{::}\KeywordTok{kable_styling}\NormalTok{(}\DataTypeTok{font_size =} \DecValTok{10}\NormalTok{) }\OperatorTok\StringTok{ }
\StringTok{  }\NormalTok{kableExtra}\OperatorTok{::}\KeywordTok{footnote}\NormalTok{(}
    \KeywordTok{paste0}\NormalTok{(}\StringTok{'Model sample size (events) is '}\NormalTok{, }
\NormalTok{           cox_multivariable_output }\OperatorTok\StringTok{ }\KeywordTok{select}\NormalTok{(}\StringTok{`}\DataTypeTok{n (events)}\StringTok{`}\NormalTok{) }\OperatorTok\StringTok{ }\KeywordTok{slice}\NormalTok{(}\DecValTok{1}\NormalTok{)))}
\end{Highlighting}
\end{Shaded}

\begin{table}[t]

\caption{\label{tab:unnamed-chunk-17}Cox Proportional-Hazards Regression Multivariable Model}
\centering
\fontsize{10}{12}\selectfont
\begin{tabular}{lllll}
\toprule
 &  & HR (95\% CI) & P Value & Overall P Value\\
\midrule
\addlinespace[0.3em]
\multicolumn{5}{l}{\textbf{Age}}\\
\hspace{1em} &  & 0.997 (0.967, 1.028) & 0.8384 & \\
\cmidrule{1-5}
\addlinespace[0.3em]
\multicolumn{5}{l}{\textbf{Gender}}\\
\hspace{1em} & Female & 1.0 (Reference) & --- & \\

\hspace{1em} & Male & 1.677 (0.863, 3.256) & 0.1269 & \\
\cmidrule{1-5}
\addlinespace[0.3em]
\multicolumn{5}{l}{\textbf{Clinical AJCC Stage}}\\
\hspace{1em} & Stage I/II (<=T2NxMx) & 1.0 (Reference) & --- & \cellcolor{yellow}{0.0349}\\

\hspace{1em} & Stage III (T3NxMx) & 2.217 (1.209, 4.065) & \cellcolor{yellow}{0.0101} & \\

\hspace{1em} & Stage IV (T4NxMx) & 1.683 (0.662, 4.275) & 0.2740 & \\
\cmidrule{1-5}
\addlinespace[0.3em]
\multicolumn{5}{l}{\textbf{Histology}}\\
\hspace{1em} & Pure Urothelial Neoplasia & 1.0 (Reference) & --- & 0.5314\\

\hspace{1em} & Mixed Tumors & 1.276 (0.678, 2.400) & 0.4501 & \\

\hspace{1em} & Variant Histology & 1.419 (0.715, 2.814) & 0.3168 & \\
\bottomrule
\multicolumn{5}{l}{\textit{Note: }}\\
\multicolumn{5}{l}{Model sample size (events) is n=166 (59)}\\
\end{tabular}
\end{table}

\hypertarget{kaplanmeier-output-functions}{%
\subsection{Kaplan--Meier Output
Functions}\label{kaplanmeier-output-functions}}

\texttt{pretty\_km\_output()} and \texttt{run\_pretty\_km\_output()} are
used to produce professional tables with Kaplan--Meier median survival
estimates, and the estimates at given time points, if listed.
\texttt{pretty\_km\_output()} uses a survfit object, while
\texttt{run\_pretty\_km\_output()} uses the variables, and strata if
applicable, and runs creates the survfit objects, also calculating the
log-rank p value, if applicable.

Many details can be adjusted, such as title (will be added as column),
strata name, confidence level, survival estimate prefix (default is
``Time''), survival estimate, median estimate, and p value rounded
digits, significant alpha level for highlighting along with color,
italic, and bolding p value options, and latex or non-latex desired
output.

\begin{Shaded}
\begin{Highlighting}[]

\CommentTok{# Using pretty_km_output() for a single comparisons}
\NormalTok{surv_obj <-}\StringTok{ }\NormalTok{survival}\OperatorTok{::}\KeywordTok{Surv}\NormalTok{(Bladder_Cancer}\OperatorTok{$}\NormalTok{Survival_Months, }
\NormalTok{                           Bladder_Cancer}\OperatorTok{$}\NormalTok{Vital_Status }\OperatorTok{==}\StringTok{ 'Dead'}\NormalTok{)   }
\NormalTok{downstage_fit <-}\StringTok{ }\NormalTok{survival}\OperatorTok{::}\KeywordTok{survfit}\NormalTok{(surv_obj }\OperatorTok{~}\StringTok{ }\NormalTok{PT0N0, }\DataTypeTok{data =}\NormalTok{ Bladder_Cancer)}

\KeywordTok{pretty_km_output}\NormalTok{(}\DataTypeTok{fit =}\NormalTok{ downstage_fit, }\DataTypeTok{time_est =} \DecValTok{60}\NormalTok{, }
                 \DataTypeTok{surv_est_prefix =} \StringTok{'Month'}\NormalTok{, }\DataTypeTok{surv_est_digits =} \DecValTok{3}\NormalTok{)}
\end{Highlighting}
\end{Shaded}

\begin{tabular}{l|l|r|r|l|l}
\hline
Group & Level & N & N Events & Median Estimate & Month:60\\
\hline
PT0N0 & No Completed Response & 131 & 57 & 48.7 (30.3, N.E.) & 0.493 (0.399, 0.609)\\
\hline
PT0N0 & Complete Response & 35 & 2 & N.E. & 0.940 (0.863, 1.000)\\
\hline
\end{tabular}

\begin{Shaded}
\begin{Highlighting}[]
\CommentTok{# Using run_pretty_km_output() for multiple comparisons}

\CommentTok{# First create vector of strata to compare (NA for no strate). }
\NormalTok{vars_to_run =}\StringTok{ }\KeywordTok{c}\NormalTok{(}\OtherTok{NA}\NormalTok{, }\StringTok{'Gender'}\NormalTok{, }\StringTok{'Clinical_Stage_Grouped'}\NormalTok{, }\StringTok{'PT0N0'}\NormalTok{, }\StringTok{'Any_Downstaging'}\NormalTok{)}

\CommentTok{# Next use purrr::map_dfr function to run the run_pretty_km_output() multiple times}
\NormalTok{km_output <-}\StringTok{ }\NormalTok{purrr}\OperatorTok{::}\KeywordTok{map_dfr}\NormalTok{(}
\NormalTok{  vars_to_run, run_pretty_km_output, }\DataTypeTok{model_data =}\NormalTok{ Bladder_Cancer, }
  \DataTypeTok{time_in =} \StringTok{'Survival_Months'}\NormalTok{, }\DataTypeTok{event_in =} \StringTok{'Vital_Status'}\NormalTok{, }\DataTypeTok{event_level =} \StringTok{'Dead'}\NormalTok{, }
  \DataTypeTok{time_est =} \KeywordTok{c}\NormalTok{(}\DecValTok{24}\NormalTok{,}\DecValTok{60}\NormalTok{), }\DataTypeTok{surv_est_prefix =} \StringTok{'Month'}\NormalTok{, }\DataTypeTok{p_digits =} \DecValTok{5}\NormalTok{, }
  \DataTypeTok{output_type =} \StringTok{'latex'}\NormalTok{) }\OperatorTok\StringTok{ }
\StringTok{  }\KeywordTok{select}\NormalTok{(Group, Level, }\KeywordTok{everything}\NormalTok{())}

\CommentTok{#Finally use kableExtra to make fancy output}
\NormalTok{kableExtra}\OperatorTok{::}\KeywordTok{kable}\NormalTok{(km_output, }\StringTok{'latex'}\NormalTok{, }\DataTypeTok{escape =}\NormalTok{ F, }\DataTypeTok{booktabs =} \OtherTok{TRUE}\NormalTok{, }\DataTypeTok{linesep =} \StringTok{''}\NormalTok{, }
     \DataTypeTok{caption =} \StringTok{'Kaplan–Meier Output (Multiple Comparisons)'}\NormalTok{) }\OperatorTok
\StringTok{  }\NormalTok{kableExtra}\OperatorTok{::}\KeywordTok{collapse_rows}\NormalTok{(}\KeywordTok{c}\NormalTok{(}\DecValTok{1}\OperatorTok{:}\DecValTok{2}\NormalTok{), }\DataTypeTok{row_group_label_position =} \StringTok{'stack'}\NormalTok{,}
                            \DataTypeTok{headers_to_remove =} \DecValTok{1}\OperatorTok{:}\DecValTok{2}\NormalTok{, }\DataTypeTok{latex_hline =} \StringTok{'major'}\NormalTok{) }\OperatorTok\StringTok{ }
\StringTok{  }\NormalTok{kableExtra}\OperatorTok{::}\KeywordTok{kable_styling}\NormalTok{(}\DataTypeTok{font_size =} \FloatTok{7.5}\NormalTok{) }\OperatorTok\StringTok{ }
\StringTok{  }\NormalTok{kableExtra}\OperatorTok{::}\KeywordTok{footnote}\NormalTok{(}\StringTok{'Survival Percentage Estimates at 24 and 60 Months'}\NormalTok{)}
\end{Highlighting}
\end{Shaded}

\begin{table}[t]

\caption{\label{tab:unnamed-chunk-19}Kaplan–Meier Output (Multiple Comparisons)}
\centering
\fontsize{7.5}{9.5}\selectfont
\begin{tabular}{llrrllll}
\toprule
 &  & N & N Events & Median Estimate & Month:24 & Month:60 & Log-Rank P\\
\midrule
\addlinespace[0.3em]
\multicolumn{8}{l}{\textbf{Overall}}\\
\hspace{1em} &  & 166 & 59 & 70.3 (48.7, N.E.) & 0.69 (0.61, 0.77) & 0.58 (0.49, 0.68) & \\
\cmidrule{1-8}
\addlinespace[0.3em]
\multicolumn{8}{l}{\textbf{Gender}}\\
\hspace{1em} & Female & 42 & 12 & N.E. (75.6, N.E.) & 0.74 (0.61, 0.89) & 0.70 (0.56, 0.87) & 0.28463\\

\hspace{1em} & Male & 124 & 47 & 68.9 (39.0, N.E.) & 0.66 (0.58, 0.76) & 0.54 (0.44, 0.66) & \\
\cmidrule{1-8}
\addlinespace[0.3em]
\multicolumn{8}{l}{\textbf{Clinical AJCC Stage}}\\
\hspace{1em} & Stage I/II (<=T2NxMx) & 118 & 34 & N.E. (61.5, N.E.) & 0.73 (0.65, 0.82) & 0.64 (0.55, 0.76) & \cellcolor{yellow}{0.04525}\\

\hspace{1em} & Stage III (T3NxMx) & 32 & 19 & 35.7 (14.3, N.E.) & 0.55 (0.40, 0.76) & 0.42 (0.27, 0.66) & \\

\hspace{1em} & Stage IV (T4NxMx) & 16 & 6 & 47.0 (11.5, N.E.) & 0.67 (0.46, 0.96) & 0.50 (0.25, 0.98) & \\
\cmidrule{1-8}
\addlinespace[0.3em]
\multicolumn{8}{l}{\textbf{Downstaged to pT0N0}}\\
\hspace{1em} & No Completed Response & 131 & 57 & 48.7 (30.3, N.E.) & 0.62 (0.53, 0.72) & 0.49 (0.40, 0.61) & \cellcolor{yellow}{0.00017}\\

\hspace{1em} & Complete Response & 35 & 2 & N.E. & 0.94 (0.86, 1.00) & 0.94 (0.86, 1.00) & \\
\cmidrule{1-8}
\addlinespace[0.3em]
\multicolumn{8}{l}{\textbf{Any Downstaging}}\\
\hspace{1em} & No Downstaging & 97 & 49 & 32.5 (21.6, 70.3) & 0.56 (0.46, 0.68) & 0.41 (0.30, 0.55) & \cellcolor{yellow}{<0.00001}\\

\hspace{1em} & Downstaging & 69 & 10 & N.E. & 0.86 (0.77, 0.95) & 0.83 (0.74, 0.94) & \\
\bottomrule
\multicolumn{8}{l}{\textit{Note: }}\\
\multicolumn{8}{l}{Survival Percentage Estimates at 24 and 60 Months}\\
\end{tabular}
\end{table}

\clearpage

\hypertarget{utility-functions}{%
\section{Utility Functions}\label{utility-functions}}

\texttt{round\_away\_0()} is a function to properly perform mathematical
rounding (i.e.~rounding away from 0 when tied), as opposed to the
\texttt{round()} function, which rounds to the nearest even number when
tied. Also \texttt{round\_away\_0()} allows for trailing zeros
(i.e.~0.100 if rounding to 3 digits).

\begin{Shaded}
\begin{Highlighting}[]
\NormalTok{vals_to_round =}\StringTok{ }\KeywordTok{c}\NormalTok{(}\OtherTok{NA}\NormalTok{,}\OperatorTok{-}\FloatTok{3.5}\OperatorTok{:}\FloatTok{3.5}\NormalTok{)}
\NormalTok{vals_to_round}
\CommentTok{#> [1]   NA -3.5 -2.5 -1.5 -0.5  0.5  1.5  2.5  3.5}
\KeywordTok{round}\NormalTok{(vals_to_round)}
\CommentTok{#> [1] NA -4 -2 -2  0  0  2  2  4}
\KeywordTok{round_away_0}\NormalTok{(vals_to_round)}
\CommentTok{#> [1] NA -4 -3 -2 -1  1  2  3  4}
\KeywordTok{round_away_0}\NormalTok{(vals_to_round, }\DataTypeTok{digits =} \DecValTok{2}\NormalTok{, }\DataTypeTok{trailing_zeros =} \OtherTok{TRUE}\NormalTok{)}
\CommentTok{#> [1] NA      "-3.50" "-2.50" "-1.50" "-0.50" "0.50"  "1.50"  "2.50"  "3.50"}
\end{Highlighting}
\end{Shaded}

\texttt{get\_session\_info()} produces reproducible tables, which are
great to add to the end of reports. The first table gives Software
Session Information and the second table gives Software Package Version
Information \texttt{get\_full\_name()} is a function used by
\texttt{get\_session\_info()} to get the user's name, based on user's
ID.

\begin{Shaded}
\begin{Highlighting}[]
\NormalTok{my_session_info <-}\StringTok{ }\KeywordTok{get_session_info}\NormalTok{()}

\NormalTok{kableExtra}\OperatorTok{::}\KeywordTok{kable}\NormalTok{(my_session_info}\OperatorTok{$}\NormalTok{platform_table, }\StringTok{'latex'}\NormalTok{, }\DataTypeTok{booktabs =} \OtherTok{TRUE}\NormalTok{, }
      \DataTypeTok{linesep =} \StringTok{''}\NormalTok{, }\DataTypeTok{caption =} \StringTok{"Reproducibility Software Session Information"}\NormalTok{) }\OperatorTok\StringTok{ }
\StringTok{      }\NormalTok{kableExtra}\OperatorTok{::}\KeywordTok{kable_styling}\NormalTok{(}\DataTypeTok{font_size =} \DecValTok{8}\NormalTok{)}
\end{Highlighting}
\end{Shaded}

\begin{table}[t]

\caption{\label{tab:unnamed-chunk-21}Reproducibility Software Session Information}
\centering
\fontsize{8}{10}\selectfont
\begin{tabular}{ll}
\toprule
name & value\\
\midrule
version & R version 3.5.2 (2018-12-20)\\
os & Windows 7 x64 SP 1\\
system & x86\_64, mingw32\\
ui & RTerm\\
language & (EN)\\
collate & C\\
ctype & English\_United States.1252\\
tz & America/New\_York\\
date & 2019-01-29\\
repo & https://github.com/wfulp/MoffittFunctions\\
file name & moffittfunctions.Rmd\\
location & vignettes\\
location & vignettes\\
user & William Fulp\\
\bottomrule
\end{tabular}
\end{table}

\begin{Shaded}
\begin{Highlighting}[]

\NormalTok{kableExtra}\OperatorTok{::}\KeywordTok{kable}\NormalTok{(my_session_info}\OperatorTok{$}\NormalTok{packages_table, }\StringTok{'latex'}\NormalTok{, }\DataTypeTok{booktabs =} \OtherTok{TRUE}\NormalTok{, }
      \DataTypeTok{linesep =} \StringTok{''}\NormalTok{, }\DataTypeTok{caption =} \StringTok{"Reproducibility Software Package Version Information"}\NormalTok{) }\OperatorTok\StringTok{ }
\StringTok{      }\NormalTok{kableExtra}\OperatorTok{::}\KeywordTok{kable_styling}\NormalTok{(}\DataTypeTok{font_size =} \DecValTok{8}\NormalTok{)}
\end{Highlighting}
\end{Shaded}

\begin{table}[t]

\caption{\label{tab:unnamed-chunk-21}Reproducibility Software Package Version Information}
\centering
\fontsize{8}{10}\selectfont
\begin{tabular}{llll}
\toprule
package & version & date & source\\
\midrule
bindrcpp & 0.2.2 & 2018-03-29 & CRAN (R 3.5.1)\\
dplyr & 0.7.8 & 2018-11-10 & CRAN (R 3.5.1)\\
MoffittFunctions & 0.1.23.1 & 2018-12-17 & local\\
\bottomrule
\end{tabular}
\end{table}


\end{document}
